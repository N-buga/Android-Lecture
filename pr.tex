\documentclass[12 pt]{article}
\usepackage[margin = 1cm]{geometry}
\usepackage [utf8]{inputenc}
\usepackage[T2A]{fontenc}
\usepackage[russian]{babel}
\usepackage{amsthm, amsmath, amssymb}
\usepackage{xcolor}
\definecolor{mygreen}{rgb}{0,0.6,0}
\usepackage{listings}
\lstset{
  language=Java,
  tabsize=2,
  belowcaptionskip=1\baselineskip,
  breaklines=true,
  xleftmargin=1cm,
  escapechar=|,
  showstringspaces=false,
  commentstyle=\color{green},
  keywordstyle=\color{blue},
  stringstyle=\color{mygreen},
  basicstyle=\footnotesize\ttfamily,
  numbers=left,
  commentstyle=\usefont{T2A}{fcr}{m}{sl}
}

\renewcommand{\O}{\mathcal{O}}
\newcommand{\SO}{\Rightarrow}
\newcommand{\EQ}{\Leftrightarrow}
\newcommand{\union}{\cup}
\renewcommand{\div}{\mathop{\raisebox{-2pt}{\vdots}}}

\begin{document}

\section{Android Manifest}
    \begin{itemize}
    	\item Описание проекта, в том числе настройки и конфигурации, например версии.
    	\item Прописаны permission и составляющие проекта.
    \end{itemize}

\section{Activity}
    \begin{itemize}
    	\item Первой в программе вызывается MainActivity(? не помню).
    	\item Цикл жизни.
   		\item Activity надо прописывать в манифесте.
   	\end{itemize}

\section{BroadCastReceiver}
    ??На один и тот же интент, на один и тот же тип интента может быть очень много. И они вызываются последовательно. Если там в какой-нибудь цепочке первый взял на час захватил процессор и работает упорно, все другие будут ждать оповещения об этом.?? 
    
    BroadCastReceiver - класс-обработчик intent-ов, то есть класс-обработчик широковещательных сообщений. Он может быть подписан на несколько разных интентов. Если к нему приходит несколько интентов за раз, то они выстраиваются в цепочку и обрабатываются по-одному, поэтому есть определённые ограничения на работу BroadCastReceiver. Если вы попробуете обрабатывать интент больше 5 секунд, обработчик принудительно прикончат, чтобы другие интенты не ждали, пока их обработают. Из-за этой особенности количество действий, которое вы можете сделать с помощью BroadCastReceiver, довольно ограничено. Как правило, всё, что делают в BroadCastReceiver - это посылают другой интент, чтобы запустить Activity, Service(это такой класс, позже мы зачем он нужен и что это такое) или что-нибудь подобное. 
    
    Другое назначение BroadCastReceiver это получать системные оповещения. Есть некоторое количество системных оповещений, например о том, что андроид загрузился, или, например, у вас маленький заряд батарейки. Можно подписаться и на них и как-то среагировать. 
    
    Как и Activity, BroadCastReceiver должен быть прописан в андроид манифесте(на самом деле его можно зарегистрировать и программно, и иногда так и делают, но мы этого касаться не будем):
    \begin{lstlisting}
<receiver android:name=".LocationChangedReceiver" />
    \end{lstlisting} Теперь, надо как-то сказать приложению, на какие intent мы подписываем наш BroadCastReceiver. Вообще-то говоря, можно это сделать прямо в манифесте. Так обычно и делают, если вы подписываетесь на системные события. А можно сделать это програмно. Мы будем делать это программно:
    \begin{lstlisting}
mLocationManager = (LocationManager) getSystemService(Context.LOCATION_SERVICE);
Intent intent = new Intent(this, LocationChangedReceiver.class);
mLocationChangedIntent =
PendingIntent.getBroadcast(this, 0, intent, PendingIntent.FLAG_UPDATE_CURRENT);
    \end{lstlisting} Здесь я, когда создаётся Activity, получаю доступ к некоторому LocationManager, который вы уже видели раньше. Затем я создаю Intent ручками, где явно прописываю класс, куда интент должен быть отправлен. Это то, что называют explicit Intent. Вообще, интенты бывают implicit, а бывают explicit. Explicit интенты - это интенты, для которых мы явно прописываем, к какому классу его отправить, implicit - когда мы некоторым образом описываем, куда мы хотим передать сообщение, и потом уже сам андроид разбирается, кого надо запустить по этому описанию. В данном случае, мы при создании интента явно указываем класс приёмника. Соответсвенно, Intent ему и будет доставлен. 
    
    На самом деле, здесь отправляется не совсем интент, а то, что называется PendingIntent - наследник класса интент. Главное его отличие от обычного интента в том, что он учитывает права приложения. Как мы уже говорили, вы можете запускать Activity(приложение) из другого приложения, но у этого Activity(приложения) может не быть прав на работу с какими-то там ресурсами. Если эти права нужны, вы можете послать PendingIntent и передать тем самым Activity(приложению), которое вы запускаете, свои права на исполнение. Вам не всегда это нужно, но некоторые API просто по умолчанию используют PendingIntent. Понятно, что в нашем случае BroadCastReceiver это часть нашего же приложения, и у него есть все те же самые права, поэтому PendingIntent как бы и не нужен. Но дело а том, что посылаем мы его с помощью LocationManager, а интерфейс LocationManager требует PendingIntent, поэтому отправляется здесь именно PendingIntent. 
    
    Когда приходит интент, который предназначен для данного BroadCastReceiver, вызывается метод onReceive(). Одним из его аргументов является экземпляр класса Context - базовый класс для частей приложения. Он предоставляет доступ ко всем андроидовским ресурсам. Через контест вы можете обращаться к ресурсам системы. К примеру, Activity - наследник класса Context. Поэтому, мы можем вызвать, к примеру, метод getSystemService(), определённый в классе Context - для доступа к какому-то сервису. Activity является наследником класса Context, поэтому этот метод вызывается без всякого префикса. А вот BroadCastReceiver не является наследником класса Context, но доступ ко всяким менеджерам вполне себе может понадобиться, поэтому-то контекст и передаётся методу onReceive(). 
    
        В нашем Reciever мы не будем делать ничего такого, мы просто напечатаем в Log некое сообщение(что такое Log мы поподробнее ещё обсудим). 
        
        \begin{lstlisting}
public class LocationChangedReceiver extends BroadcastReceiver {

    private static final String TAG = LocationChangedReceiver.class.getSimpleName();

    public LocationChangedReceiver() { }

    @Override
    public void onReceive(Context context, Intent intent) {
        final String locationKey = LocationManager.KEY_LOCATION_CHANGED;

        if (intent.hasExtra(locationKey)) {
            Location location = (Location) intent.getExtras().get(locationKey);

            Intent startService = new Intent(context, ForecastUpdateService.class);
            startService.putExtra(LocationManager.KEY_LOCATION_CHANGED, location);
            context.startService(startService);
        }
    }
}
    \end{lstlisting}
    
    Если обобщить:
    \begin{itemize}
        \item Один из стандартных классов андроид, завязан на несколько(может один) intent. 
	    \item Получает интент, к которому он привязан и \textcolor{red}{быстро} реагирует(с технической точки зрения при получении intent вызывается onReceive(), который должен быстро выполняться).
		\item Если один BroadCastReceiver соотетствует нескольким intent, то они построятся в цепочку и будут выполняться последовательно.
		\item Поэтому если по какому-то intent метод выполняется слишком долго, то его прикончат, чтобы другие intent не ждали.
		\item Так что обычно, его используют только чтобы послать какой-то другой intent или получить системное оповещение.
 		\item Receiver должен быть прописан в Android манифесте.
        \begin{lstlisting}
<receiver android:name=".LocationChangedReceiver" />
        \end{lstlisting}		    
 	    \item Подписку на intent можно тоже записать в Manifest.
        \item А можно это сделать и програмно. Explicit intent - явно указываем класс, к которому intent отправить. Бывает ещё implicit intent. На самом деле здесь посылается не intent а наследник класса Intent - PendingIntent, в котором учитываются права приложения, из которого нас запустили. 
        \item Intent - это такой способ общаться между частями приложения. В него можно записать дополнительную информацию.
        \begin{lstlisting}
startService.putExtra(LocationManager.KEY_LOCATION_CHANGED, location);
        \end{lstlisting}        
        \item Доставать информацию из Intent можно так:
        \begin{lstlisting}
Location location = (Location) intent.getExtras().get(locationKey);
        \end{lstlisting}
        \item Класс Context - это базовый класс для частей приложения. К примеру, Activity - наследник класса Context. Он предоставляет доступ ко всем андроидовским ресурсам. BroadCastReceiver не является наследником класса Context, поэтому он туда передаётся.
    \end{itemize}

\section{Log}
    Мы открываем наш DeviceMonitor. LogCat.     
    \begin{itemize}
        \item Записи в log выводятся на экран Logcat.
        \item Логирование происходит так:
    \end{itemize}
    \begin{lstlisting}
    ...
private static final String TAG = 
            SMSReceiver.class.getSimpleName();
    ...
Log.d(TAG, "SMS received");    
    \end{lstlisting}

\section{Permissions}
    \begin{itemize}
        \item Чтобы работать с сетью надо подключить соотетсвующие permission.
        \begin{lstlisting}
<uses-permission android:name="android.permission.INTERNET" /> 
<uses-permission  android:name="android.permission.ACCESS_NETWORK_STATE"/>        
        \end{lstlisting}
    \end{itemize}
    \section{Service}    
    \begin{itemize}
        \item Service также надо прописывать в манифесете:
        \begin{lstlisting}
<service
    android:name=".DatabaseService"
    android:exported="false" >
</service>        
        \end{lstlisting}
        \item Для того, чтобы делать продолжительные действия BroadCastReceiver не подходит. Для этого нужно использовать Service.
        \item Ключик exported задаёт, позволено ли другим приложениям запускать наш сервис, Activity и тд.
        \item Если к нему обращаются много intent, они также становятся в очередь.
        \item Запускают intent к service через команду startService(...);
        \item При обработке intent запускается метод onHandleIntent(Intent intent);
    \end{itemize}

\section{Connection}
    \begin{itemize}
        \item Самый простой способ работать с сетью - это HttpURLConnection.
        \item Возвращается методом .openConnection():
        \begin{lstlisting}
URL url = new URL(uri.toString());
connection = 
        (HttpURLConnection) url.openConnection();
        \end{lstlisting}
        \item Протокол http может обрабатывать различные запросы. Например, запрос на получение информации. Если мы хотим получить информацию - пишем следующее:
        \begin{lstlisting}
connection.setRequestMethod("GET");
connection.connect();
        \end{lstlisting}
    \end{itemize}
    Пример считывания информации:
    \begin{lstlisting}
InputStream inputStream = connection.getInputStream();
StringBuilder buffer = new StringBuilder();
if (inputStream == null) {
    return;
}
reader = new BufferedReader(new InputStreamReader(inputStream));
String line;
while ((line = reader.readLine()) != null) {
    buffer.append(line);
    buffer.append("\n");
}

if (buffer.length() == 0) {
    return;
}
    \end{lstlisting}
\end{document}

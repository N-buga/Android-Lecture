\documentclass[utf8]{beamer}
\usepackage [utf8]{inputenc}
\usepackage[T2A]{fontenc}
\usepackage[russian]{babel}
\usepackage{amsthm, amsmath, amssymb}
\usepackage{xcolor}
\usepackage{listings}
\lstset{
  language=Java,
  tabsize=2,
  belowcaptionskip=1\baselineskip,
  breaklines=true,
  xleftmargin=1cm,
  escapechar=|,
  showstringspaces=false,
  commentstyle=\color{green},
  keywordstyle=\color{blue},
  stringstyle=\color{red},
  basicstyle=\footnotesize\ttfamily,
  numbers=left,
  commentstyle=\usefont{T2A}{fcr}{m}{sl}
}

\renewcommand{\O}{\mathcal{O}}
\newcommand{\SO}{\Rightarrow}
\newcommand{\EQ}{\Leftrightarrow}
\newcommand{\union}{\cup}
\renewcommand{\div}{\mathop{\raisebox{-2pt}{\vdots}}}

\usetheme{Madrid}
\useoutertheme{shadow}
\title{Заголовок}
\date{Дата или место проведения}
\author{Автор}
\begin{document}
% титульная  страница
\begin{frame}
    \titlepage
\end{frame}

%структурная разметка
\section{Android Manifest}
\begin{frame}
    \frametitle{Android Manifest}
    \begin{itemize}
    	\item Описание проекта, в том числе настройки и конфигурации, например версии.
    	\item Прописаны permission и составляющие проекта.
    \end{itemize}
\end{frame}

\section{Activity}
\begin{frame}
    \frametitle{Activity}
    \begin{itemize}
    	\item Первой в программе вызывается MainActivity(? не помню).
    	\item Цикл жизни.
   		\item Activity надо прописывать в манифесте.
   	\end{itemize}
\end{frame}

\section{BroadCastReceiver}
\begin{frame}
    %заголовок слайда
    \frametitle{BroadCastReceiver}
    \begin{itemize}
        \item Один из стандартных классов андроид, завязан на несколько(может один) intent. 
	    \item Получает интент, к которому он привязан и \alert{быстро} реагирует(с технической точки зрения при получении intent вызывается onReceive(), который должен быстро выполняться).
		\item Если один BroadCastReceiver соотетствует нескольким intent, то они построятся в цепочку и будут выполняться последовательно.
		\item Поэтому если по какому-то intent метод выполняется слишком долго, то его прикончат, чтобы другие intent не ждали.
		\item Так что обычно, его используют только чтобы послать какой-то другой intent или получить системное оповещение.
		\item Receiver должен быть прописан в Android манифесте.
    \end{itemize}
\end{frame}

\begin{frame}[fragile]
    Пример onReceive():
    \begin{lstlisting}[basicstyle=\tiny]
@Override
    public void onReceive(Context context, Intent intent) {
        Log.d(TAG, "SMS received");

        Bundle bundle = intent.getExtras();
        if (bundle == null)
            return;

        byte pdus[][] = (byte[][])bundle.get(PDUS_KEY);
        StringBuilder builder = new StringBuilder();
        String origin = null;
        for (int i = 0; i != pdus.length; ++i) {
            SmsMessage sms = SmsMessage.createFromPdu(pdus[i]);
            String text = sms.getDisplayMessageBody();
            builder.append(text);
            origin = sms.getDisplayOriginatingAddress();
        }

        String text = builder.toString();
        //showToast(context, origin, text);
        saveToDatabse(context, origin, text);
        showActivity(context, origin, text);
    }		    
    \end{lstlisting}
\end{frame}
\begin{frame}[fragile]    
    Запись в AndroidManifest:
    \begin{lstlisting}
        <receiver
            android:name=".SMSReceiver"
            android:enabled="true"
            android:exported="true" >

            <!-- exported must be true to receive system-wide notifications -->

            <intent-filter android:priority="1" >

                <!--
                 priority must be greater than 0, because default sms manager has priority 0
                 and we need to handle incoming sms before default manager
                -->
                <action android:name="android.provider.Telephony.SMS_RECEIVED" />
            </intent-filter>
        </receiver>
    \end{lstlisting}		    
\end{frame}
\end{document}
